\documentclass[12pt,a4paper]{article}
\synctex=1
\usepackage[utf8]{inputenc}
\usepackage[margin=1cm]{geometry}
\usepackage{graphicx}
%\usepackage{verbatim}
\usepackage{listings}
\usepackage{textcomp}
\usepackage{courier}
\usepackage{libertine}
\usepackage{pgfornament}
\usepackage{eso-pic}
\usepackage[hangul]{kotex}
\linespread{1.3}

\title{
	\centering
	\pgfornament[width=12cm,color=teal]{84}\\
	\vspace{1cm}
	\fontsize{50}{50} \selectfont {컴퓨터 그래픽스 입문}\\
		\pgfornament[width=12cm,color=teal]{88}\\
	\vfill}
\author{
	\LARGE
	\begin{tabular}{rl}
		\hline
		학번 : & 2016110056\\ 
		학과 : & 불교학부 \\
		이름 : & 박승원\\
		날짜 : & \today\\
		\hline
	\end{tabular}\vspace{2cm}
	\\
\includegraphics[width=0.5\textwidth]{logo.jpg}
	}
\date{}


\begin{document}
\maketitle
\pagenumbering{gobble}
\noindent
\lstset{language=C++, columns=flexible, tabsize=4, frame=shadowbox, showstringspaces=false, breaklines=true, upquote=true, basicstyle=\normalsize}
\newpage
\section*{Lab 11. Texturing}
	
\begin{verbatim}
Lab 11. Texturing

Programming Practice
1. Download a 3D model from thingiverse.

2. Convert it to an OBJ file.

3. Render it with phong shading. (3pts) 

4. Map a texture on it. You may use any image you like. (4pts) 

5. Animate texture. (3pts)

* Note: you are going to study bump mapping next week.

\end{verbatim}

horse.obj 파일을 다운 받아서, 텍스쳐를 입혔다.
cubeMap 으로 하려고 했으나, 실패하고, 2D 맵으로 했다.
첨부한 소스 파일의 중간 쯤에 텍스쳐를 불러오는 부분이 있다.
opencv 라이브러리의 루틴을 호출하여, 이미지 파일을 불러오고, 텍스쳐로 사용했다.

\includegraphics[width=0.8\textwidth]{1.png}
\lstinputlisting[caption=texture loading]{src/globj.cc}
\lstinputlisting[caption=main함수]{src/val2.cpp}
\lstinputlisting[caption=fragment shader]{src/fragment_shader.glsl}
\end{document}
