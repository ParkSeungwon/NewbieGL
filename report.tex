\documentclass[12pt,a4paper]{article}
\synctex=1
\usepackage[utf8]{inputenc}
\usepackage[margin=1cm]{geometry}
\usepackage{graphicx}
%\usepackage{verbatim}
\usepackage{listings}
\usepackage{textcomp}
\usepackage{courier}
\usepackage{libertine}
\usepackage{pgfornament}
\usepackage{eso-pic}
\usepackage[hangul]{kotex}
\linespread{1.3}

\title{
	\centering
	\pgfornament[width=12cm,color=teal]{84}\\
	\vspace{1cm}
	\fontsize{50}{50} \selectfont {컴퓨터 그래픽스 입문}\\
		\pgfornament[width=12cm,color=teal]{88}\\
	\vfill}
\author{
	\LARGE
	\begin{tabular}{rl}
		\hline
		학번 : & 2016110056\\ 
		학과 : & 불교학부 \\
		이름 : & 박승원\\
		날짜 : & \today\\
		\hline
	\end{tabular}\vspace{2cm}
	\\
\includegraphics[width=0.5\textwidth]{logo.jpg}
	}
\date{}


\begin{document}
\maketitle
\pagenumbering{gobble}
\noindent
\lstset{language=C++, columns=flexible, tabsize=4, frame=shadowbox, showstringspaces=false, breaklines=true, upquote=true, basicstyle=\normalsize}


1. Draw a smiley as pretty as you can. You may google 'smiley face' for more an better references.(5)\\
Scoring guide\\
- Draw a color filled face : 1 point\\
- Draw a smiley mouth : 1 point\\
- Draw a eyes(left + right) : 2 point\\
- How much pretty smiley face  : 1 point\\

2. Move the smiley by up/down/left/right key inputs. (5)\\
Scoring guide\\
- move left : 1 point\\
- move right : 1 point\\
- move up : 1 point\\
- move down : 1 point\\
- how well did you write the report : 1 point\\

원을 그리기 위한 클래스로 원위의 점들을 벡터로 가진다.
\lstinputlisting{src/circle.h}

기본 매트릭스, 위치 이동과 기본 자료형으로 쓰임.
\lstinputlisting{src/matrix.h}

메인 파일
\lstinputlisting{src/glfw2.cpp}	
	
\end{document}
