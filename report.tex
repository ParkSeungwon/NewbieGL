\documentclass[12pt,a4paper]{article}
\synctex=1
\usepackage[utf8]{inputenc}
\usepackage[margin=1cm]{geometry}
\usepackage{graphicx}
%\usepackage{verbatim}
\usepackage{listings}
\usepackage{textcomp}
\usepackage{courier}
\usepackage{libertine}
\usepackage{pgfornament}
\usepackage{eso-pic}
\usepackage[hangul]{kotex}
\linespread{1.3}

\title{
	\centering
	\pgfornament[width=12cm,color=teal]{84}\\
	\vspace{1cm}
	\fontsize{50}{50} \selectfont {컴퓨터 그래픽스 입문}\\
		\pgfornament[width=12cm,color=teal]{88}\\
	\vfill}
\author{
	\LARGE
	\begin{tabular}{rl}
		\hline
		학번 : & 2016110056\\ 
		학과 : & 불교학부 \\
		이름 : & 박승원\\
		날짜 : & \today\\
		\hline
	\end{tabular}\vspace{2cm}
	\\
\includegraphics[width=0.5\textwidth]{logo.jpg}
	}
\date{}


\begin{document}
\maketitle
\pagenumbering{gobble}
\noindent
\lstset{language=C++, columns=flexible, tabsize=4, frame=shadowbox, showstringspaces=false, breaklines=true, upquote=true, basicstyle=\normalsize}

\section{Geometric Transformation}
Programming Practice\\
(Write down what you did and how you made the result. Otherwise I don’t know what you did.)\\
1. Change the size of the box using scale.(3pt)

2. Translate the position of the box.(3pt)

3. Revolve the box around the origin.(2pt)

4. Add the box that is revolved along previous box. (1pt) 

5. The added box revolves and rotates like earth. (1pt)

\includegraphics[width=\textwidth]{1.png}
\lstinputlisting{src/val.cpp}	
소스 코드에 주석을 참조바랍니다.
	
\end{document}
